%%\documentclass[onecolumn, draftclsnofoot,10pt, compsoc]{IEEEtran}
\documentclass[journal,10pt,onecolumn,compsoc]{IEEEtran} 
\usepackage[margin=1.0in]{geometry} 
%%\geometry{textheight=9.5in, textwidth=7in}
\usepackage{pdfpages}
%\usepackage{graphicx}
\usepackage{caption,graphicx,float} 
\usepackage{listings}
\usepackage{verbatim}
\usepackage[english]{babel}
\usepackage{url}
\usepackage{setspace}
%\usepackage{tabu}
%\setlength{\parskip}{\baselineskip} 
\setlength\parindent{24pt}

% 1. Fill in these details
\def \CapstoneTeamName{			HART CS Capstone}
\def \CapstoneTeamNumber{		11}
\def \GroupMemberOne{			Rick Menzel}
\def \GroupMemberTwo{			Matthew Forsland}
\def \CapstoneProjectName{		High Altitude Rocketry Project}
\def \CapstoneSponsorCompany{		OSU American Institute of Aeronautics and Astronautics (AIAA)}
\def \CapstoneSponsorPerson{		Dr. Nancy Squires}

% 2. Uncomment the appropriate line below so that the document type works
\def \DocType{			%Problem Statement
				%Requirements Document
				%Technology Review
				%Design Document
				Progress Report
				}
			
\newcommand{\NameSigPair}[1]{\par
\makebox[2.75in][r]{#1} \hfil 	\makebox[3.25in]{\makebox[2.25in]{\hrulefill} \hfill		\makebox[.75in]{\hrulefill}}
\par\vspace{-12pt} \textit{\tiny\noindent
\makebox[2.75in]{} \hfil		\makebox[3.25in]{\makebox[2.25in][r]{Signature} \hfill	\makebox[.75in][r]{Date}}}}
% 3. If the document is not to be signed, uncomment the RENEWcommand below
%\renewcommand{\NameSigPair}[1]{#1}

%%%%%%%%%%%%%%%%%%%%%%%%%%%%%%%%%%%%%%%
\begin{document}
\begin{titlepage}
    \pagenumbering{gobble}
    \begin{singlespace}
    	\includegraphics[height=4cm]{coe_v_spot1}
        \hfill 
        % 4. If you have a logo, use this includegraphics command to put it on the coversheet.
        %\includegraphics[height=4cm]{CompanyLogo}   
        \par\vspace{.2in}
        \centering
        \scshape{
            \huge CS Capstone \DocType \par
            {\large\today}\par
            \vspace{.5in}
            \textbf{\Huge\CapstoneProjectName}\par
            \vfill
            {\large Prepared for}\par
            \Huge \CapstoneSponsorCompany\par
            \vspace{5pt}
            {\Large\NameSigPair{\CapstoneSponsorPerson}\par}
            {\large Prepared by }\par
            Group\CapstoneTeamNumber\par
            % 5. comment out the line below this one if you do not wish to name your team
            \CapstoneTeamName\par 
            \vspace{5pt}
            {\Large
                \NameSigPair{\GroupMemberOne}\par
                \NameSigPair{\GroupMemberTwo}\par
            }
            \vspace{20pt}
        }
        \begin{abstract}
        % 6. Fill in your abstract
		
        	%This allows you to have sensible diffs when you use \LaTeX with version control, as well as giving a quick visual test to see if sentences are too short/long.
        	%If you have questions, ``The Not So Short Guide to LaTeX'' is a great resource (\url{https://tobi.oetiker.ch/lshort/lshort.pdf})
        \end{abstract}     
    \end{singlespace}
\end{titlepage}
\newpage
\pagenumbering{arabic}
\setlength{\parskip}{\baselineskip}
\tableofcontents
% 7. uncomment this (if applicable). Consider adding a page break.
%\listoffigures
%\listoftables
\clearpage

%\setlength{\parskip}{\baselineskip} 
% 8. now you write!

\section{Purpose \& Goals}

	\subsection{Purpose}
		\noindent The primary purpose of this project is to produce software for the processing, storage and visualization of avionics data during rocket launches.
		HART is a multidisciplinary team, and the CS capstone team is working closely with ME and ECE teams who have responsibility for the actual production, transmission and reception of rocket telemetry.
		Once received at ground level, we will perform processing on the data, including the derivation of values such as thrust and (at least above certain altitudes) altitude.
		Processed data is then written to non-volatile storage for preservation, and is broadcast over Wi-Fi for display on wireless equipped devices.
		The storage is important both to ensure the rocket performs as expected during tests and to allow future teams to learn from this year's project.
		The visualization allows team members to monitor for unsafe conditions, track (and recover) the rocket stages, and, at least in a way, to fly with the rocket they have built over the course of 9+ months.
	
	\subsection{Goals}
		\noindent The primary goal of the CS capstone team during Fall term has been to research the challenges of capturing telemetry from a high-altitude, high-speed rocket and subsequently work with our client to plan a suitable software solution.
		In physical form, this has been represented by several documents: a Problem Statement, a Requirements Document, a series of Technology Reviews, and, finally, a Design Document, with each document essentially building on the previous.
		Going forward, the team is prepared to begin the implementation of the solution we have designed.
		Beginning in Winter term, HART as a whole is moving into a production phase.
		As rocket components come together, the intention is to stage a series of flight tests progressing up to a potential full-scale, full-range flight some time in May 2019.
		These flights serve not only as tests of the rocket hardware, but as critical opportunities to test the avionics systems, including those developed by this team.
		And, much like a hardware failure can lead to time or cost overruns during this stage of the project, issues with the software systems can lead to the loss of valuable test data.
		Finally, all of HART's efforts will come together for a final competition launch at the Spaceport America Cup in June 2019.
\newpage
%-------------------------------------------------------------------------------------------

\section{Weekly Summaries}

	\subsection{Week 4}
		
		\subsubsection{Activities}
			During this week, the HART CS capstone was formed and assigned to the project.
			Activities completed were thus largely related to setup and introductions.
			On the first evening after being assigned, the CS team attended their first HART weekly meeting.
			During this meeting, team members met each other for the first time and further introduced themselves to the rest of HART and to the client.
			A team Github repo was established and access granted to team members, the assigned TA (Behnam Saeedi) and both course instructors.
			Weekly meeting times for HART, the CS team and the assigned TA were set, and a team member attended the Requirements Document info session during office hours.
			Finally, individual team members completed drafts of the Problem Statement which were subsequently combined into a combined team Problem Statement.
		
		\subsubsection{Problems}
			The primary challenge during this early stage was a variable degree of LaTeX familiarity within the team. 
			Fortunately, however, the more experienced team member was able to share their expertise and the required assignments were completed on schedule.
	
	\subsection{Week 5}
	
		\subsubsection{Activities}
			During this week, the team met with the client, Dr. Squires, to discuss the forthcoming Customer Requirements document.
			Following this meeting, team members completed the document and received (favorable) feedback from the client.
			Several tasks relating to the team budget were completed, though this was a formality as the CS team is not using any team funds.
			The CS sub-team further assisted HART with their universal customer requirements.
			Weekly team, sub-team, and TA meetings were attended.
			
		\subsubsection{Problems}
			Following feedback on the Problem Statement, it became apparent that the provided/required LaTeX template was not in fact accurate.
			Additional research was required regarding manual formatting of LaTeX documents IAW IEEE standards.
			An attempt to communicate with the client over email regarding the Requirements Document produced no response, requiring the team to make contact in-person.
		
	\subsection{Week 6}
	
		\subsubsection{Activities}
			Following customer feedback on the Requirements Document, the team moved on to the completion of individual Tech Reviews.
			This led directly to the breakdown of the project into two halves, with Matt Forsland taking responsibility for the back-end or data-processing portion, and Rick Menzel taking responsibility for the front-end, or visualization portion.
			Conveniently, this task division aligned well with each team-member's respective interests.
			Additionally, an underclassman joined HART this week with an interest in helping with the back-end portion of the project, he was briefed on the way capstone works and given some research-related tasks to get acquainted.
			Weekly team, sub-team, and TA meetings were attended.
				
		\subsubsection{Problems}
			An attempt to meet with the client during her scheduled office hours led to the discovery that this is impractical due to very long wait times. 
			As such, much of the future communication with Dr. Squires has been completed during weekly HART meetings (which she generally attends).
			Further issues appeared regarding LaTeX, specifically a gremlin relating to the inclusion of a reference page.
			Some tweaks were made to the working template as a result.
	
	\subsection{Week 7}
	
		\subsubsection{Activities}
			Personal final-draft Tech Reviews were completed by both team members.
			The team met with Dr. Squires another time regarding the work completed thus far and the plan moving forward.
			Notably, this discussion largely cemented the decision to implement a web-based visualization solution.
			Weekly team, sub-team, and TA meetings were attended.
			
		\subsubsection{Problems}
			No issues to report.
	
	\subsection{Week 8}
	
		\subsubsection{Activities}
			After receiving grades/feedback on previous assignments, the team began some revision work.
			Both team members continued research into potential implementation details.
			One team member participated in the AIAA trip to Blue Origin's production facility/HQ in Kent, WA.
			Weekly team, sub-team, and TA meetings were attended.
		
		\subsubsection{Problems}
			One team member had a schedule conflict with the TA meeting, though alibis were made to both Ben and Matt prior to the scheduled meeting time.
	
	\subsection{Week 9}
		
		\subsubsection{Activities}
			During this week, the team completed the Design Document building off all of the work thus far.
			Weekly team, sub-team, and TA meetings were attended.
		
		\subsubsection{Problems}
			The availability of additional details regarding the SA Cup in June led one team member to realize a potential logistical conflict.
			This was discussed with the team client and Kirsten.
			Rick Menzel may not attend the SA Cup, but if so will ensure Matt Forsland is able to perform any necessary duties at that time.
	
	\subsection{Week 10}
		
		\subsubsection{Activities}
			The team submitted the Design Document and began work on the end-of-term Progress Update.
			After noticing some inconsistencies between the laws regarding the use of commercial GPS systems on rockets and the information previously given to the team, some research was conducted.
			This led to discussions with ME and ECE avionics team members in the interest of having GPS limitations removed from team instruments by the relevant manufacturers.
			Further revisions to previous documentation including Customer Requirements.
			Weekly team, sub-team, and TA meetings were attended.
		
		\subsubsection{Problems}
			There were several issues related to the use of Kaltura.
			Notably, the recording of a joint team video produced an unusable product due to lag and inconsistent audio in the exported video.
			Subsequent attempts produced videos that play back as expected in Kaltura, but have high degrees of lag/de-syncing when exported.
			As such, team members recorded audio for the Progress Report separately.
\newpage
%-------------------------------------------------------------------------------------------

\section{Retrospective}

\begin{center}
\begin{table}[h!]
\begin{tabular}{|p{0.3\linewidth}|p{0.3\linewidth}|p{0.3\linewidth}|} 
	\hline
	\textbf{Positives} & \textbf{Deltas} & \textbf{Actions} \\
	\hline
	Scheduled weekly meetings offer a convenient time for communication with client and other sub-teams & Meeting with client during office hours incurs significant wait times & Reserve client meetings for during team meetings when possible \\ 
	\hline
	HART Slack channel allows for easy, recorded communication between team members & Use of mixed contact methods led to minor delays in communication & Rely on Slack channel moving forward \\ 
	\hline
	TA provided excellent example documents - would have been lost with it! & Unclear team understanding of laws governing GPS use led to lack of clarity regarding the usefulness of GPS during flight  & Research current laws regarding the operation of GPS units at high speeds/altitudes \\
	\hline
	Excellent opportunities to meet with industry representatives including from SpaceX and Blue origin & Inconsistencies in provided LaTeX template led to lack of clarity regarding document formatting & Use TA-provided examples more for formatting going forward \\
	\hline
\end{tabular}
\caption{Fall Term Retrospective}
\label{table:1}
\end{table}
\end{center}
\newpage
%-------------------------------------------------------------------------------------------

\section{State of the Project}
	As of this writing, the HART CS capstone team has produced a preliminary Design Document and is prepared to move forward with implementation.
	In conjunction with Dr. Squires and the remainder of HART, the team has arrived at a web-based design whereby the processing and storage of data is conducted by the ground station hardware prior to broadcasting it via Wi-Fi for visualization.
	The design of the final system has been subdivided into a back- and front-end and responsibility has been delegated between team members.
	
	\noindent Preliminary work has been completed regarding the 2D component of the visualization system, as well as on the data-processing component of the system.
	This has both increased team-member understanding of the path forward and informed the remainder of the system design.
	Some preliminary conceptualization work has been done regarding the 3D component of the visualization system.
	
	\noindent Leading into Winter Term, it is this team's intention to begin prototyping work on the complete system.
	The goal is to have a Minimum Viable Prototype ready for work close to the start of next term, allowing the team to focus more on refinements, bug-fixes and client feedback than otherwise possible.
	

\end{document}