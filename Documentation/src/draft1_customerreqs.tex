\documentclass[onecolumn, draftclsnofoot,10pt, compsoc]{IEEEtran}
\usepackage{graphicx}
\usepackage{url}
\usepackage{setspace}

\usepackage{geometry}
\geometry{textheight=9.5in, textwidth=7in}

% 1. Fill in these details
\def \CapstoneTeamName{			HART CS Capstone}
\def \CapstoneTeamNumber{		11}
\def \GroupMemberOne{			Rick Menzel}
\def \GroupMemberTwo{			Matthew Forsland}
\def \CapstoneProjectName{		High Altitude Rocketry Project}
\def \CapstoneSponsorCompany{	OSU American Institute of Aeronautics and Astronautics (AIAA)}
\def \CapstoneSponsorPerson{	Dr. Nancy Squires}

% 2. Uncomment the appropriate line below so that the document type works
\def \DocType{		Requirements Document
				%Requirements Document
				%Technology Review
				%Design Document
				%Progress Report
				}
			
\newcommand{\NameSigPair}[1]{\par
\makebox[2.75in][r]{#1} \hfil 	\makebox[3.25in]{\makebox[2.25in]{\hrulefill} \hfill		\makebox[.75in]{\hrulefill}}
\par\vspace{-12pt} \textit{\tiny\noindent
\makebox[2.75in]{} \hfil		\makebox[3.25in]{\makebox[2.25in][r]{Signature} \hfill	\makebox[.75in][r]{Date}}}}
% 3. If the document is not to be signed, uncomment the RENEWcommand below
%\renewcommand{\NameSigPair}[1]{#1}

%%%%%%%%%%%%%%%%%%%%%%%%%%%%%%%%%%%%%%%
\begin{document}
	\begin{titlepage}
		\pagenumbering{gobble}
		\begin{singlespace}
			\includegraphics[height=4cm]{coe_v_spot1}
			\hfill 
			% 4. If you have a logo, use this includegraphics command to put it on the coversheet.
			%\includegraphics[height=4cm]{CompanyLogo}   
			\par\vspace{.2in}
			\centering
			\scshape{
				\huge CS Capstone \DocType \par
				{\large\today}\par
				\vspace{.5in}
				\textbf{\Huge\CapstoneProjectName}\par
				\vfill
				{\large Prepared for}\par
				\Huge \CapstoneSponsorCompany\par
				\vspace{5pt}
				{\Large\NameSigPair{\CapstoneSponsorPerson}\par}
				{\large Prepared by }\par
				Group\CapstoneTeamNumber\par
				% 5. comment out the line below this one if you do not wish to name your team
				\CapstoneTeamName\par 
				\vspace{5pt}
				{\Large
					\NameSigPair{\GroupMemberOne}\par
					\NameSigPair{\GroupMemberTwo}\par
				}
				\vspace{20pt}
			}
			\begin{abstract}
			% 6. Fill in your abstract
				ABSTRACT
				%This document is written using one sentence per line.
				%This allows you to have sensible diffs when you use \LaTeX with version control, as well as giving a quick visual test to see if sentences are too short/long.
				%If you have questions, ``The Not So Short Guide to LaTeX'' is a great resource (\url{https://tobi.oetiker.ch/lshort/lshort.pdf})
			\end{abstract}     
		\end{singlespace}
	\end{titlepage}

	\newpage
	\pagenumbering{arabic}
	\tableofcontents
	% 7. uncomment this (if applicable). Consider adding a page break.
	%\listoffigures
	%\listoftables
	\clearpage


	% 8. now you write!
	\section{Introduction}
		[Description of the section]

		\subsection{Purpose}
			The purpose of this document is to provide a detailed description of the requirements for the HART CS groundstation software. 
			It will describe the purpose and scope of the project and the specific requirements of the completed system. 
			It will also detail the interface and interactions with the HART ECE avionics system.

		\subsection{Scope}
			The HART CS groundstation is a system which receives telemetry data transmitted wirelessly from two remote systems located on separate stages of 2-stage rocket. 
			It will interpret the raw telemetry data to calculate derived values, and provide a web application which visualizes all data in 2-dimensional graphs and provides a 3-dimensional visualization of the rockets' trajectories.

		\subsection{Glossary}
			\begin{itemize}
				\item \textbf{Avionics}
					The guidance system for the rocket, which controls engine ignition and parachute deployment.
					The avionics system also transmits telemetry to a groundstation.

			\end{itemize}

		\subsection{References}
			[Remove if unnecessary]

		\subsection{Overview}
			The following sections provide a description of the necessary system functions and of system interactions with external libraries and external data sources.
			[Description of each section]


	\newpage
	\section{Overall Description}
		[Description of the section]

		\subsection{Product Perspective}
			This software will consist of a server backend which receives telemetry data broadcast by avionics systems located on both stages of a 2-stage rocket, and interprets that data into additional derived values. 
			The server backend will communicate with a web application which provides visualization for raw and interpreted data in 2-dimensional graphs, and generates a 3-dimensional model for the trajectory of both stages of the rocket.

		\subsection{Product Functions}
			[Basic usage of the software]

		\subsection{User Characteristics}
			The types of users who will use this software fall into two main categories. 
			The first type of user includes personnel operating the rocket, who require the groundstation to accurately receive telemetry broadcasted by the rocket in real time, and provide an accurate location of the rocket once landed to facilitate quick recovery of the rocket.
			The operators will also need access to the received telemetry to analyze the performance of the rocket.
			The second type of user includes observers using the web application to monitor the condition of the rocket during flight.

		\subsection{Constraints}
			[Qualification and cause of constraints]
			The software must receive telemetry data broadcast in realtime.
			The software must store all received and interpreted data for later use.
			The graphical user interface must be reliable and dependable.

		\subsection{Assumptions and Dependencies}
			[Assumptions]

			\noindent The software is dependent upon the HART ECE avionics system being operational and able to broadcast telemetry for any tests during which the groundstation software is used.
			The software is dependent upon the groundstation being able to reliably receive the telemetry broadcast by the rocket without interference.

		\subsection{Apportioning of Requirements}
			[Input desired]
			In the case that the project is delayed, some requirements involving the web application can be dismissed. Provided that telemetry data is stored on the groundstation, the 2-dimensional visualization can be performed by third-party software without the use of a specialized web-application.


	\newpage
	\section{Specific Requirements}
		[Description of the section]

		\subsection{Interfaces}
			[Describe the web application? I think?]

		\subsection{Functional Requirements}
			This section includes the requirements that specify the fundamental actions of the software system. Furthermore, it describes the necessary outcomes that are expected by the client.

			\noindent
			\textbf{ID: FR1}\\
			TITLE: Velocity Interpretation\\
			DESC: The software should interpret the acceleration and altitude telemetry into velocity data.\\
			RAT: The telemetry data will provide only information pertaining to the rocket's orientation, acceleration, altitude, and GPS position.
				Determining the velocity of the rocket allows users to correlate changes in performance with the passing of specific velocities, such as Mach one.\\
			DEPEND: \\
			
			\noindent
			\textbf{ID: FR2}\\
			TITLE: Thrust Interpretation\\
			DESC: The software should interpret the acceleration telementry into the thrust provided by the rocket motors.\\
			RAT: Providing the thrust curve produced by a rocket motor allows users to analyze the rocket motor's performance.\\
			DEPEND: \\
			
			[Here's a template for this, go ahead and write whatever requirements you can think of]
			\noindent
			\textbf{ID: FR[N]}\\
			TITLE: \\
			DESC: \\
			RAT: \\
			DEPEND: \\
		
		\subsection{Performance Requirements}
			This section provides a detailed specification of user interaction with the software and performance metrics for the software.
		
			\noindent
			\textbf{ID: QR1}\\
			TITLE: Telemetry Reception\\
			DESC: The software should receive telemetry in realtime.\\
			RAT: The avionics systems will be broadcasting telemetry in realtime. 
				If the groundstation software is not receiving data at the same rate, packets of telemetry data will be lost.\\
			DEPEND: \\
			
			\noindent
			\textbf{ID: QR[N]}\\
			TITLE: Realtime Visualization\\
			DESC: The web application should update with telemetry data in realtime, as it is received by the groundstation.\\
			RAT: Monitoring the rocket's status while it is in flight but not visually tracked is important in determining if the rocket is falling at terminal velocity, which poses a significant safety hazard to personnel on the ground. \\
			DEPEND: QR1\\

			[Here's a template for this, go ahead and write whatever requirements you can think of]
			\noindent
			\textbf{ID: QR[N]}\\
			TITLE: \\
			DESC: \\
			RAT: \\
			DEPEND: \\

\end{document}